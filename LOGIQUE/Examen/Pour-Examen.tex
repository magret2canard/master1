\documentclass{article}
\usepackage[frenchb]{babel}
\usepackage[utf8]{inputenc}
\usepackage{lettrine}
\usepackage{fullpage}
\usepackage{amsmath}
\usepackage{amsthm}
\usepackage{array}
\usepackage{proof}
%%%%%%%%%%%%%%%%%%%%%%%%%%%%%%%%%%%%%%%%%
\title{Préparation à la partie "Coq" de l'examen d'INF 471}
%\author{Jeremy Mourguet, Mathieu Picaud}
%\date{31 Octobre 2008}
%%%%%%%%%%%%%%%%%%%%%%%%%%%%%%%%%%%%%%%%%
%\begin{proof}[Proof of the Main Theorem]
%\begin{proof}
%\end{proof}
%%%%%%%%%%%%%%%%%%%%%%%%%%%%%%%%%%%%%%%%%
\renewcommand{\proofname}{Preuve}
\newtheorem{preuve}{Preuve}


\begin{document}

\maketitle


 Il ne vous sera évidemment pas demandé de donner des preuves complètes en syntaxe Coq (sans l'aide d'un assistant à la démonstration), mais de :
 \begin{enumerate}
\item savoir écrire des preuves simples (manipulation des connecteurs et quantificateurs  +  égalité)
\item savoir quel schéma d'induction appliquer dans quelle situation, et expliquer  quels sont les sous-buts engendrés.
\item On ne demande pas de preuve complète, mais une indication des étapes principales.
\end{enumerate}
    
\noindent On ne demande pas de résoudre complètement des sous-buts triviaux, mais d'expliquer  pourquoi ils sont faciles à résoudre :\\

\noindent \underline{exemple :}\\

Soit le but :
\begin{verbatim}
    A:Type
    P : A -> Prop
    R : A -> A -> Prop
    a : A
    b : A
    H : forall x:A, (exists y:A, R x y) -> P x
    H0 : R a b
     ------------------------
     P a
\end{verbatim}

\noindent L'explication suivante est tout à fait satisfaisante : "on applique H, ce qui engendre un sous-but d'énoncé  "exists y:A, R a y",  ce sous-but est résolu par la règle d'introduction du quantificateur existentiel, en  donnant b comme témoin". Si on est sûr de soi, on peut aussi répondre par une suite d'étapes :
\begin{verbatim}
   apply H.

  .... même contexte que ci-dessus
  ------------------
  exists y: A, R a y

   exists b;assumption.
   Qed.
\end{verbatim}

\noindent Des questions porteront sur l'arithmétique de Peano . Le fichier Given.v  ci-joint contient --- outre les définitions et lemmes vus en projet --- une définition de la notion de parité (Even) et d'imparité (Odd), plus quelques lemmes (certains ont une preuve incomplète, afin de vous permettre de vous exercer sur cette formalisation). On vous demandera d'expliquer comment prouver des résultats simples (qui n'apparaissent pas dans le fichier joint) sur la parité ou l'imparité.

\end{document}